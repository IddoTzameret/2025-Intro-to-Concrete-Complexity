%%%%%%%%%%%%%%%%%%%%% chapter2.tex %%%%%%%%%%%%%%%%%%%%%%%%%%%%%%%%%
%
%  Monotone Circuits Lower bounds 
%
% Use this file as a template for your own input.
%
%%%%%%%%%%%%%%%%%%%%%%%% Springer-Verlag %%%%%%%%%%%%%%%%%%%%%%%%%%
%\motto{Use the template \emph{chapter.tex} to style the various elements of your chapter content.}





\chapter{Monotone Circuit Lower Bounds}
\label{sec:Razborov} % Always give a unique label
% use \chaptermark{}
% to alter or adjust the chapter heading in the running head


% Always give a unique label
% and use \ref{<label>} for cross-references
% and \cite{<label>} for bibliographic references
% use \sectionmark{}
% to alter or adjust the section heading in the running head

We've seen that proving that SAT is not in \Ppoly, i.e., can't be solved by polynomial-size circuits, implies that $\P \neq \NP$.
Due to the notorious difficulty of these questions, we are interested in proving \emph{weaker} lower bounds, namely, some lower bounds against restricted classes of circuits. 
Here, we study such a restricted circuit class: A boolean circuit without negation gates, i.e., monotone circuits.

\begin{definition}[Monotone circuit]
A Monotone circuit is a Boolean circuit that contains fan-in two gates AND and OR, but has \emph{no} NOT gates.
\end{definition}

This means in particular that monotone circuits can compute only monotone functions: a Boolean function is said to be monotone if increasing the number of ones in the input cannot flip the value of the function from 1 to 0. 

More precisely, for $\bar{x}, \bar{y} \in\{0,1\}^n$, write $\bar{x} \geqslant \bar{y}$ iff $ \forall i \in [n], x_i \geqslant y_i$, where $[n]$ denotes $\{1,\dots,n\}$. (Here, $x_i\ge y_i$ for Boolean $x_i,y_i$ means simply that $1\ge 0$ and $0\ge 0$, $1\ge 1$, while $0\not\ge 1$.)

\begin{definition}[Monotone function]
A Boolean function $f:\{0,1\}^n \rightarrow\{0,1\}$ is said to be  \emph{monotone} if $\forall \bar{x} \geq \bar{y}, f(\bar{x}) \geqslant f(y)$.
\end{definition}


Many NP problems are monotone, like CLIQUE:

Given an undirected graph $G=(V, E)$ with $n$ nodes, a $k$-clique in $G$ is a set $U\subseteq V$ of size $k$, st. every pair of nodes $u_1, u_2 \in U$ is connected by an edge (in $E$):

$$
 \forall u_1 \in u \forall u_2 \in u ( u \neq u_2\Rightarrow (u_1, u_2)\in E).
$$


Recall that a computational (decision) problem is a \emph{language}, namely an infinite set of finite strings over a finite alphabet (usually the alphabet $\{0,1\}$). Here, our language consists of all the strings that encode (in some natural way) an accepted graph, i.e., a $k$-clique with $n$ nodes.
The natural way to encode a graph in our case is this: a graph  $G=(V, E) $ with $n$ nodes, is encoded by $\binom{n}{2}$ input variables  $x_{ij}$, where the semantic of the encoding is: $x_{i j}=1$ iff $(i, j) \in E$. In other words, if the input variable $x_{ij}=1$,   our input graph contains the edge $(i,j)$, and otherwise it does not. 

We are interested in CLIQUE$(k, n)$ for a fixed $k$, as the following Boolean function: 
\begin{svgraybox}
The computational problem \textbf{CLIQUE$(k, n)$}: 

\textit{Input}: Undirected graph $G=(V,E)$ with $n$ nodes, and a number $k$ (given in unary, i.e., $1^k$).

\textit{Accept}: if the graph $G$ contains a $k$-clique. 

\textit{Reject}: otherwise.
\end{svgraybox}


 Note that $\operatorname{CLIQUE}(k, n)$ is a monotone function: If we add 1's to the input, we only \emph{increase} the chance it has a $k$-clique!
 
It is known that CLIQUE$(k,n)$ is NP-complete (see standard complexity textbooks; e.g., Papadimitriou 1994).


Since CLIQUE$(n, k)$ is a monotone (Boolean)
function we can compute it by a monotone Boolean circuit.

\begin{trailer}{Example of a monotone circuit computing CLIQUE$(n, k)$}
 ``Run" over all $\binom{n}{k} ~~ k$-sub-graphs in $G$, and check if at least one of those is a clique:

\begin{figure}
    \centering
    \includegraphics[width=0.75\linewidth]{images/k-clique-simple-circuit.png}
    \label{fig:enter-label}
\end{figure}

$S_1, S_2, \ldots, S_{\binom{n}{k}}$ are the $\binom{n}{k}$ subgraphs in $G$ each of size $k$.
Size of this circuit: $O\left(k^2 \cdot\binom{n}{k}\right)$.

\end{trailer}


\begin{svgraybox}
\textbf{Crude-circuits $CC\left(X_1, \ldots, X_m\right)$ for CLIQUE}.
A  circuit for computing $\operatorname{CLIQUE}  (n, k),$ consisting of a big $\lor$ of all $S_i$'s, each computed by $\land$'s of edges, as in the example above, is called a \textbf{\textit{crude-circuit}} for CLIQUE($n, k)$.

We denote by $CC\left(S_1, \ldots, S_{\binom{n}{k}}\right)$  the crude circuit computing the $\lor$ of all subgraphs $S_1 \ldots S_{n\choose k }$. In general, we can use different subgraphs of different size $X_i$ in a crude-circuit: $CC\left(X_1, \ldots, X_m\right)$ for $X_i \subseteq V$, where the $X_i$'s have different size, possibly different from $k$.
\end{svgraybox}

\begin{note} When $k=w(\log n)$, $CC\left(S_1, \ldots, S_{\left(k_k\right)}\right)$ is of exponential size.
\end{note}

The following theorem shows this naive monotone circuit cannot be improved much:

\begin{figure}
    \centering
    \includegraphics[width=0.3\linewidth]{images/RAZBOROV_Alexander.jpeg}
    \caption{Alexander Razborov. Creator: Kozloff, Robert | Credit: Photo by Robert Kozloff.
Copyright: The University of Chicago}
    \label{fig:enter-label}
\end{figure}

\begin{svgraybox}
\begin{theorem}[Razborov \cite{Razb85}]\label{thm:razborov} 
Let $k=\sqrt[4]{n}$. Then, every monotone circuit computing CLIQUE$(n, k)$ has size $2^{\Omega(\sqrt[8]{n})}$.
\end{theorem}
\end{svgraybox}
That is, exists a constant $c$ s.t. for large enough $n \in \mathbb{N}$, if $C_n$ computes CLIQUE$(n,k)$ then $\left|C_n\right| \geqslant 2^{c \cdot \sqrt[8]{n}}$.


The rest of this chapter aims to prove this theorem. Our exposition is taken from Papadimitriou's textbook \cite{Pap94}.


%Recall: a crude circuit is a big OR of cliques, each computed as a big AND.



\paragraph{Approximation Method}

We first provide an overview of the approach we take to prove \Cref{thm:razborov} which is called \emph{the approximation method}. 
We shall describe a way of approximating any \textbf{monotone} circuit for $\operatorname{CLIQUE}({n}, {k})$ by a crude circuit, namely a big OR of cliques, as follows.

\begin{enumerate}
    
\item  Given a monotone circuit C , we shall construct a crude circuit ${CC}({X} 1, \ldots, {Xm})$ for some m and $\left|{X}_{{i}}\right| \leq l$ (for some $l$, all ${i}=1, \ldots, {m}$ ), that approximates $\operatorname{CLIQUE}({n}, {k})$ with \textbf{precision} that is dependent on the number of gates in $C$.

\item I.e., if the precision is not good, namely the crude circuit ${CC}({X} 1, \ldots, {Xm})$ for CLIQUE (n,k) we end up with makes \textit{many errors} on the CLIQUE ( $n, k$ ) function (i.e., says "NO" on an input that has a k-clique, and "YES" on an input that has no k-clique), it means that the circuit $C$ has \textit{many gates}, and vice versa.\label{it:approximation-b}

\item We show that every crude circuit ${CC}({X}_1, \ldots, {X_m})$ ($|{X}_i| \leq l$ for some $l$, for all $i=1, \ldots, m$ ), ought to make \textit{exponentially many errors} on the function $\operatorname{CLIQUE}(n, k)$. From \ref{it:approximation-b} above we conclude that the number of gates in C was exponential.

\end{enumerate}


The \textbf{approximation} (i.e., construction of a crude circuit for CLIQUE(${n},{k})$ given the circuit $C$) will proceed in steps, one step for each gate of the monotone circuit:

\begin{enumerate}
    
\item If $C$ is a monotone circuit computing $\operatorname{CLIQUE}({n}, {k})$ we can \textit{approximate} any gate OR or AND in $C$ with a crude circuit.

\item Each such approximation step introduces rather few errors (false positives and false negatives).
\end{enumerate}


\section{Proof of monotone circuit lower bounds}

\begin{trailer}{Parameters \& notation}

Recall we want to compute CLIQUE$(n, k)$
with $n$ the number of nodes in the graph and $k$ the size of a clique within the graph. 
We set:
$$
k=\sqrt[4]{n}.
$$

\begin{svgraybox}\textbf{Goal}: Show that every monotone circuit computing $\operatorname{CLIQUE}(n,k)$ has size at least $2^{c \sqrt[8]{n}}$ for some constant $c$ (for sufficiently large $n$).
\end{svgraybox}

$$
\begin{aligned}
& l=\sqrt[8]{n} \\
& p \approx \sqrt[8]{n} \\
& M=(p-1)^l \cdot l! & \approx(\sqrt[8]{n}-1)^{\sqrt[8]{x}} \cdot(\sqrt[8]{n})! \\
& & \leq(\sqrt[4]{n})^{\sqrt[8]{n}}
\end{aligned}
$$


Each crude-circuit we use in the approximation is:

$$
C C\left(x_1, \ldots, x_m\right)
$$

for $m \leqslant M$ and $\left|X_i\right| \leqslant l, \forall i \in[m]$.

\end{trailer}

\bigskip 





- The approximation of the monotone circuit $C$ that computes $\operatorname{CLIQUE}(n, k)$ is done by induction on the size of $C$, ie., number of $\lor, \wedge$ gates in $C$.


- Comment: Such induction is also called ``Induction on the structure of $C$''.

- Such induction proceeds as follows: 


\para{Base Case} $|C|=1$, ie., $C$ consists of only a single input gate $g_{ij}$. Recall $g_{i j}$ is an input gate denoting whether $(i, j) \in E$, for $i, j \in V$.
That is, if there is an edge between $i$ and $j$ in the input graph $G$.

This is an easy case: We need to show a crude cat $\operatorname{CC}\left(X_1, \ldots, X_m\right)$ with $m \leqslant M$ and $\left|X_i\right| \leqslant l \quad \forall i \in[m]$ that
approximates $g_{i i}$ (without introducing too many errors; We shall count precisely the number of potential errors later).
But the circuit $C C(\{i, j\})=g_{i j}$ by definition. (Hence, no errors here!)


\para{Induction Step}
Given two crude circuits
$\operatorname{CC}(\mathcal X)$ and 
$\operatorname{CC}(\mathcal Y) 
$, with $\mathcal{X}=\{X_1, \ldots, X_m\}$, 
$\mathcal{Y}=\{Y_1, \ldots, Y_{m^{\prime}}\}$, $m\le M$, and $\left|X_i\right| \leq \ell $, for all $i$, $m'\le M $ and $\left|Y_i\right| \leq \ell$, for all $i$.


We wish to construct another crude circuit  for computing $CC(\mathcal{X}) \vee CC(\mathcal{Y})$, and $\operatorname{CC}(\mathcal X) \wedge 
\operatorname{CC}(\mathcal Y)$.

\case{1}
$\lor$-gate.

\textit{Naive attempt}: $\operatorname{CC}(\mathcal X)
\lor \operatorname{CC}(\mathcal Y)$ is approximated by $\operatorname{CC}(\mathcal X \cup \mathcal Y)$. That is, $\operatorname{CC}\left(X_1, \ldots, X_m, Y_1, \ldots, Y_m\right)$. At first glance this is a good solution because it does not introduce any errors (why?). But there is a \textit{problem}: what if $m+m^{\prime}>M$?


\textit{Solution}: We need to cleverly \emph{reduce} the number of sets $X_1, \ldots, X_m, Y_1, \ldots, Y_{m^{\prime}}$. To do this we use a combinatorial lemma called The Sunflower Lemma.





\section{The Sunflower Lemma}

\begin{figure}
    \centering
        \includegraphics[width=0.75\linewidth]{images/sunflower-lemma.png}
    \caption{From https://theorydish.blog/2021/05/19/entropy-estimation-via-two-chains-streamlining-the-proof-of-the-sunflower-lemma/ by Lunjia Hu}
    \label{fig:enter-label}
\end{figure}



%Let $U$ be some universe, namely a set of elements (e.g., nodes). Let $Z=\{Z_1,\dots,Z_M\}$ be a family of sets from the universe, i.e., $Z_i\subseteq U$, for each $i$, with $M$ some natural number. %We call the family $P$ a \emph{sunflower} if  $\left\{P_1, \ldots, P_p\right\}$ called \emph{petals}, each $\left|P_i\right| \leqslant \ell$ where $\ell=\sqrt[8]{n}$, such that all pairs $P_i \neq P_j$ in the family share the \emph{same} intersection, called the \emph{core} of the sunflower. In other words, there is a set  $ P_i\cap P_j = Core$

\begin{svgraybox}
\begin{definition}[Sunflower] Let $U$ be some universe, namely a set of elements (e.g., nodes). Let $P=\{P_1,\dots,P_p\}$ be a family of distinct sets from the universe, i.e., $P_i\subseteq U$, for each $i$, with $p$ some natural number. 
We call the family $P$ a \emph{sunflower} if each all pairs $P_i \neq P_j$ in the family $P$ share the \emph{same} intersection, called the \emph{core} of the sunflower.
In other words, there is a (possibly empty) set $\mathrm{core}\subseteq U$, such that for all $i\neq j$,  $ P_i\cap P_j = \mathrm{core}$.
If $P$ is a sunflower we call the $P_i$'s the \emph{petals} of the sunflower $P$.
\end{definition}
\end{svgraybox}
Note: It's okay if the core is the empty-set! This means all petals are (pairwise) disjoint.


\begin{svgraybox}
\textbf{Sunflower Lemma} (Erd\"os-Rado): For every $\ell, p$, let $Z$ be a family of more than $M=(p-1)^\ell \cdot \ell!$ non-empty sets each of size $\leqslant \ell$ (over some universe $U$). Then, $Z$ contains a sunflower of size $p$. In other words, $Z$ contains $p$ sets $\left\{P_1, \ldots, P_p\right\}$, each $P_i$ has size $\leqslant \ell$, and the intersection of every pair $P_i \neq P_j$ is fixed: $P_i \cap P_j=P_{i'} \cap P_{j'}$, for all $i \neq j \neq i' \neq j^{\prime}$.
\end{svgraybox}



\begin{proof}[Proof of the Sunflower Lemma]
By induction on $\ell$.

\Base  $\ell=1$. Thus the statement we need to show is that $p$ different singletons form a sunflower. Which is true (the core is $\varnothing$ ).


\induction $\ell>1$. Consider a family $D \subseteq Z$ of pairwise disjoint sets that is maximal in the sense that if we add any new set in $Z$ to the family $D$, the sets in $D$ are not pairwise disjoint anymore. That is, every set in $Z \backslash D$ intersects some set in $D$.

\case 1 If $D$ contains $\geqslant p$ sets then $D$ is a sunflower with empty core, and we are done.

\case 2 Otherwise, let $E$ be the \emph{union} of all sets in $D$.
Since $|D|<p$, i.e., $D$ contains less than $p$ sets, and each set in $D$ has size $\le \ell$, we know that $|E| \leqslant(p-1)\cd\ell$.

\medskip 

Moreover, $E$ intersects every set in $Z$ by assumption.
Since $Z$ has more than $M$ sets by assumption, and each set intersects some element of $E$, there exists an element $d \in E$ that intersects $>\frac{M}{(p-1) l}=(p-1)^{l-1} \cdot(l-1)!$ sets in $Z$.


\begin{remark}
If a set $E$ intersect all sets in a family of sets $X_1, . ., X_M$ then there is an element in $E$ that appears in $\geqslant \frac{M}{|E|}$ sets $X_i$.
Otherwise, each element in $E$ appears in $<\frac{M}{|E|}$ sets $X_i$. Thus, $E$ intersects $<|E| \cdot \frac{M}{|E|}=M$ sets $X_i$, which is a contradiction to the assumption.
\end{remark}


Consider

$$
Z^{\prime}:=\{z \backslash\{d\} \mid z \in Z \text { and } d \in z\} .
$$


We know that $Z^{\prime}$ contains more than $M^{\prime}=(p-1)^{l-1} \cdot(l-1)!$ sets.
By \emph{induction hypothesis} (since $M^{\prime}$ is "$M$ with $\ell$ decreased by one"), $Z^{\prime}$ contains a sunflower denoted $\left\{P_1, \ldots, P_p\right\}$ with $\left|p_i\right| \leq \ell-1$, for all $i$.
Hence, $\left\{P_1 \cup\{d\}, \ldots, P_p \cup\{d\}\right\}$ is a sunflower in $Z$.
This concludes the Sunflower Lemma's proof.
\end{proof}



Approximating $O R$ and AND using Plucking
- By the Sunflower Lemma, every family of $\geqslant M$ nonempty sets, each of cardinality $\leqslant l$, then we can find a sunflower in it with $l=\sqrt[3]{n}(p \approx \sqrt[3]{n}) M=(p-1)^l \cdot \ell!$.
- Plucking a sunflower: replacing all petals by their core. 4 sets

Corollary: If we have $>M$ sets in a family, by repeated plucking we can reduce the number of sets to $\leq M$ (if we cant apply plucking. anymore we know by the Sunflower Lemma that the number of sets is $\leqslant M$ ).
pluck (z): the result of repeated plucking of a family of sets $Z$, until $|Z| \leqslant M$.
Definition (Approximate OR and AND): foxily of sets The approximate $-V$ of two crude -circuits $C C(x)$ and $\operatorname{cc}(y)$ is: $\operatorname{cc}($ pluck $(x \cup y))$.

The approximate $-\Lambda$ of $C C(x)$ and $C C(Y)$ is: $\operatorname{cc}\left(\right.$ pluck $\left\{X_i \cup Y_j:\left|X_i \cup Y_j\right| \leqslant l\right.$ and $\left.\left.X_i \in \chi, Y_j \in Y\right\}\right)$





False Positives \& False Negatives

We shall now show that $V$ and $\Lambda$-approximators are good, in the sense that they introduce few errors!

Out of all possible inputs to a cat computing Clique ( $n, k$ ) (for $k=\sqrt[4]{n})$, the ne are:
Accept-instances: $G=(V, E)$ is a graph on $n$ modes that contains a $k$-clique.
Reject-instances: $G=(V, E)$ is a graph on $n$ nodes that does not contain a $k$-clique.
We shall restrict attention $t_0$ only subsets of Accept and Rejed instances. This will be sufficient for the proof (it's more convenient that way). Thus, our focus is on "extreme" cases of inputs:



- Positive-lnputs: $G=(v, E)$ has $x$ nodes and a $k$-clique ( $k$ nodes $w /$ all edges between them); while no other edge exist in $G$.


\includegraphics{images/clique1.png}



There are $\binom{x}{k}$ positive-inputs, each one is dearly an accept input to $\operatorname{CUQ} \cup E(n, k)$.
- Negative Inputs: $G=\left(v_1 E\right)$ has $n$ nodes and $(k-1)$-colouring (independent sets). In other words, nodes $V$ are coloured by some $k-1$ colours (one colour for each independent set [ie., set of nodes w/ no edge between them]). And all edges between notes w/ different colours.



\includegraphics{images/clique2.png}


Note: 1) There are $(k-1)^n$ negative -inputs. (We count twice two identical graphs w/ colours interchanged.)
2) A $(k-1)$-colouring is a negative-input for $\operatorname{CUQUE}(n, k)$ because a $k$-clique cannot be coloured by $k-1$ colours.
3) In fact, even a single edge added to a $(k-1)$-colouring will make the graph contain a k-clique!




$V$-approximator of $\operatorname{CC}(x) \vee \operatorname{CC}(y)$ introduces
1) a false negative:

A k-clique $G=(v, E)$ s.t.: $(\operatorname{cc}(x) \operatorname{vcc}(y))(G)=1$
2) a false positive:
$\operatorname{cedpluck}(x \cup y)(G)=0$
$V$-aterosimator
$A(k-1)$-coburing $G=(v, E)$ s.t.

$$
\begin{aligned}
& (\operatorname{cc}(x) \vee \operatorname{cc}(y))(G)=0 \\
& \frac{\operatorname{cc}(p l u c k(x \cup y))(G)}{V \text {-approvinator }}=1
\end{aligned}
$$


1 -apporimator of $\operatorname{CC}(x) \wedge \operatorname{CC}(y)$ introduces
1) a false negative:

A k-clique $G=(v, E)$ s.t: : $(\operatorname{cc}(x) \wedge \operatorname{cc}(y))(G)=1$
2) a false positive: $c c\left(P \mid\right.$ luck $\left.\left\{x_i\left|y_i:\left|x_i \cup y_i\right| \leq l, x_i \in X, y_i \in\right\}\right\}\right)=0$

$$
\left.c c\left(p l u c k\left\{x_i \cup Y_j:\left|x_i \cup y_0\right| \leq l, x_i \in \chi, y_i \in\right\}\right\}\right)=1
$$


Lemma 1: Each approximation step introduces at most $M^2 \cdot \frac{(k-1)^n}{2^p}$ false positives.

Lemma 2: Each approximation step introduces at most $M^2 \cdot\binom{n-l-1}{k-l-1}$ false negatives.

Proof of monotone Pkt lower bound from these lemmas. CLAIM: Every crude-ctt CC $\left(X_1, \ldots, x_m\right) w /$ $\left|x_i\right| \leq l$ and $m \leqslant M$ is either identically 0 (and thus is wrong on all positive-instances), or outputs 1 on at least half of the negative -instances.
Proof: If $\operatorname{CC}\left(x_1 \ldots x_m\right) \neq 0$, then it accept's at least those graphs $G$ that have cliques on at least one of the sets $X_i$, for some $i$. The size of $X_i$ is $\leqslant l<k$, thus many graphs $G$ that are not $k$-cliques still have e-cliqnes



cont.
Let's count how many such graphs exist, using probability. Consider a graph $G=(V, E)$; assign randomly (and independently) from among $(k-1)$ colours to the modes of $G$. Let $v_1 \neq v_2 \in V$ be two nodes.

$$
\operatorname{Pr}\left[\text { colour of } v_1=\text { colour of } v_2\right]=\frac{1}{k-1}
$$


Denote by $R\left(X_i\right)$ the evert that in the random colouring to $G$, there exists a pair of nodes with the same colour in $X_i$
Then, $\operatorname{Pr}\left[R\left(x_i\right)\right] \leqslant \frac{\binom{x_i \mid}{ 2}}{k-1} \leqslant \frac{\binom{l}{2}}{k-1} \leqslant \frac{1}{2}$.
there are $\binom{\mid x_i i}{2}$ pairs
of nodes in $X_i$. We use
the union bound.
This means that out of all negative-instances to CLIQUE $(n, k)$ (namely, all ( $k-1)$-colouring of a graph w/ $n$ nodes), at least half of them are going to colour $X_i$ w/ different colours. Hence, for these negative -instance, there will be edges between all nodes in $X_i$, and thus $\left(C\left(x_1, \ldots, x_m\right)(G)=1\right.$ for each of these negative instances $G$.




We are now ready to conclude the main result.
Recall: $l=\sqrt[8]{n}$ and
Let $p=\sqrt[8]{n} \cdot \log n$. Thus, $M_i=(p-1)^l \cdot l!<\left(n^{\frac{1}{3}}\right)^{\sqrt[8]{n}}$, for large
- Let $C$ be a monotone cat computing CLQUE $(n, \sqrt[4]{n})$.
- We apply the approximators at each gate of $C$ iteratively.
- The output gate of $C$ thus is written as a $\operatorname{CC}\left(x_1, \ldots, x_m\right)$ for some $m \leqslant M$ and $\left|x_i\right| \leqslant l$.
- Based on the above, we have two cases:

Case 1: $\operatorname{cc}\left(x_1, \ldots, x_m\right) \equiv 0$.
Thus, the number of false negatives introduced is the total of all possible positive-instance, namely all possible $k-c$ cliques: $\binom{n}{k}$.
Thus,

$$
|C| \geqslant \frac{\binom{n}{k}}{\operatorname{lemmax}^2\binom{n-l-1}{k-l-1}} \geqslant \frac{1}{M^2}\left(\frac{n-l}{k}\right)^l \geqslant n^{c \sqrt[8]{n}},
$$


Case $2: \operatorname{CC}\left(x_1, \ldots, x_n\right)$ has at least $\frac{1}{2}(k-1)^n$ false positives (half of $(k-1)$-colourings).
By Lemma 2: $|c| \geqslant \frac{\frac{1}{2}(k-1)^n}{M^2 \frac{(k-1)^n}{2^p}}=\frac{2^p}{2 \cdot M^2}>n^{c \cdot 8} n$, for $c=\frac{1}{3}$.



Lemma 1: Each approximation step introduces at most $M^2 \frac{(k-1)^n}{2^p}$ false positives.

Pe of F: Case 1: OR-approximator
We start with $\overline{C C}\left(x_1, \ldots, x_m\right)$ and $\operatorname{CC}\left(y_1, \ldots, y_{m_m}\right)$ and consider a false positive introduced by

$$
\operatorname{cc}\left(\text { pluck }\left(X_1, \ldots, X_m, Y_1, \ldots, Y_{m^{\prime}}\right)\right) \text {. }
$$


That is, a $G=(v, E)$ s.t.,

$$
\operatorname{cc}\left(x_1, \ldots, x_m\right)(G)=0, \quad \operatorname{cc}\left(y_1, \ldots, y_m\right)(G)=0
$$

and

$$
\operatorname{cc}\left(\operatorname{pluck}\left(X_1, \ldots, X_m, Y_1, \ldots, Y_{m^{\prime}}\right)\right)(G)=1 .
$$


We consider each plucking involved in (2), and bound from above the number of false positive introduced by this plucking. (Note this is the only reason a false positive can be introduced.)


 - Consider a single plucking : replace sunflower $\left\{z_1, \ldots, z_p\right\}$ by its core $Z$.

By (1): $Y_i^{\prime} s \& X_i^{\prime} s(\&, ' s)$
are all hon-cliques
Thus, by (2)
$Z$ is a clique
and every petal $z_i$
has two nodes w/ the $\frac{\text { same colour! }}{L \operatorname{bg}(1)}$



\includegraphics{images/clique3.png}






- We count the number of such colourings.
- We do this probabilistically:

Choosing a $(k-1)$-colouring of the nodes randomly and independently, what's the probability every $z_i$ has repeated colours but $z$ does not?
- As before, let $R(X)$ be the probability that $X$ has repeated colours.


$\qquad$ As before, let $R(X)$ be the probability that $X$ has repeated colours.

$$
\operatorname{pr}\left[R\left(z_1\right) \wedge \ldots \wedge R\left(z_p\right) \wedge \neg R(z)\right] \leqslant \operatorname{pr}\left[R\left(z_1\right) \wedge \ldots \wedge R\left(z_p\right) \mid \neg R(z)\right]$$

$$
\begin{aligned}
& =\prod_{i=1}^p \operatorname{Pr}\left[R\left(z_i\right) \mid \neg R(z)\right] \\
& \leqslant \prod_{i=1}^p \operatorname{Pr}\left[R\left(z_i\right)\right]
\leqslant \frac{1}{2^p}
\end{aligned}
$$

$$
\operatorname{Pr}[A \| B]=\operatorname{Pr}[A \mid B] \cdot \operatorname{Pr}[B]
$$


Li's don't have common nodes, except those in $Z$.

We've seen before that

$$
\operatorname{Pr}[R(x)] \leqslant \frac{1}{2}
$$

Probability of repetition of colours is increased if we doit restrict ourselves to colourings w/ no repetitions in $z \subseteq Z_i$.


Finally, since the approximation step entails up to $\frac{2 M}{p-1}$ pluckings (each plucking decreases the number of sets by $\mathrm{p}-1$, and there are no more than 2 M sets when we start), the lemma holds for the OR approximation step: because e

$$
M^2 \cdot \frac{(k-1)^n}{2^p} \geqslant \frac{2 M}{p-1} \cdot \frac{(k-1)^n}{2^p}
$$


Cont. of proof of Lemma 1

Consider now an AND approximation step of crude circuits $C C(x)$ and $C(y)$. It can be broken down in three phases: First, we form $C C(\{X \cup Y: X \in \chi, Y \in Y\})$; this introduces no false positives, because any graph in which $X U Y$ is a clique must have a clique in both and $Y$, and thus it was accepted by both constituent crude circuits. The second phase omits from the approximator circuit several sets (those of cardinality larger than $\ell$ ), and can therefore introduce no false positives. The third phase entails a sequence of fewer than $\mathrm{M}^2$ plucking, during each of which, by the analysis of the OR case above, at most $2^{-\mathrm{P}}(\mathrm{k}-1)^{\mathrm{n}}$ false positives are introduced. The proof of the lemma is complete: because at total we introduced $\leq M^2 \cdot \frac{(k-1)^n}{2^p}$ false positives




Lemma 2: Each approximation step introduces $\leqslant M^2\binom{n-l-1}{k-l-1}$ false negatives.
PRoOF:
Case $V_{:}$a false negative:
A $k$-clique $G=(v, E)$ st.: $(\operatorname{cc}(x) \vee \operatorname{ccc}(y))(G)=1$ (

$$
\underbrace{\operatorname{colpluck}(x \cup y)(G)}_{V \text {-aproximator }}=0
$$


This is impossible, because plucking only deletes sets and make them smaller; hence if (1) holds then (2) cannot.
Case 1:
A false negative:
A $k$-clique $G=(v, E)$ st: : $(\operatorname{cc}(x) \wedge \operatorname{cc}(y))(G)=1$

$$
\left.c\left(P / \mid u c k\left\{x_i \cup Y_j:\left|x_i \cup Y_i\right| \leq l, x_i \in \chi_{,}, Y_i \in\right\}\right\}\right)=0 \text { (2) }
$$


In the first stop we replace $c a(x) \wedge c(y)$ by $C C(\{X \cup Y: x \in X, y \in Y\})$.
Hence, if $G$ is a $K$-clique and both $X$ and $Y$ are each cliques in $G$, it must be that $X \cup Y$ is also a clique in $G(w h y$ ? ). Thus, no false negatives are introduced in this step.


$\frac{\text { (cont.) }}{\text { We next delete all sets } \overbrace{X_i \cup Y_j}^{\text {denoted }} Z}$ larger than $l$. This introduce several false negatives:
All $k$-cliques $G$ that contain $Z$.
We calculate an upper bound on these false negatives: There are precisely $\binom{n-|z|}{k-|z|}$ k-cliques er that contain $z$ (as part of the clique). Since, $|z|>l$, the upper bound on the false negatives introduced by each deletion is $\binom{n-l-1}{k-l-1}$.

Since there are $\leqslant M^2$ deletions of sets, (because $|\chi|=|y|=M$ ), we get that the number of false negatives introduced by $a_n 1$-approximation step is $\leqslant M^2 \cdot\binom{n-t-1}{k-l-1}$.
